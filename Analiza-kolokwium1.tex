\documentclass[a5paper,12pt]{article}
\usepackage{polski}
\usepackage{graphicx}
\usepackage[utf8]{inputenc}
\usepackage{geometry}
\newgeometry{tmargin=0.5cm, bmargin=0.5cm, lmargin=1.5cm, rmargin=1.5cm}
\usepackage{amsmath}
\usepackage{amsfonts}
\usepackage{amssymb}
\usepackage{textcomp}


%opening
\title{Kolokwium nr 1 \\z Analizy Matematycznej}
\author{dr Alina Gleska \\(w zastępstwie za dr. Leszka Wittenbecka)}

\begin{document}

\maketitle

\begin{abstract}
Analiza Matematyczna, Kolokwium nr 1 (w. B), Informatyka (grupa I5) 2019/2020, PP.
\end{abstract}
\begin{enumerate}
 \item Obliczyć granice funkcji: \\
    $\lim_{x\rightarrow \infty}x(\sqrt{x^2+1}-x)$, 
    $\lim_{x\rightarrow 0^+}sin(x)(ctg(2x))^2$
 \item Wyznaczyć pochodne funkcji: \\
    $f(x)=(ln(\frac{x^2-x+1}{x}))^3$, $g(x)=(cosx)^{2sinx} $
 \item Obliczyć granice funkcji: \\
 $\lim_{x \rightarrow 0^+}(arctg(x))^x$, $\lim_{x\rightarrow \infty}\frac{ln(5+3^x)}{ln(7+2^x)}$
 \item Wyznaczyc wszystkie asymptoty funkcji: \\
 $f(x)=\frac{x}{e^x-1}$
 \item Zbadac monotoniczność funkcji i wyznaczyc ekstrema: \\
 $f(x)=x^2\sqrt{36-x^2}$
 \item Zbadać przebieg zmienności funkcji: \\
 $f(x)=\frac{x}{lnx}$ (wraz z narysowaniem wykresu)
\end{enumerate}
\end{document}
