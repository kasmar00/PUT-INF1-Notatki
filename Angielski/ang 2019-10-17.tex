\documentclass[a4paper,10pt]{article}
\usepackage[utf8]{inputenc}
\usepackage[T1]{fontenc}
\usepackage[MeX]{polski}


%opening
\title{Angielski 02}
\author{}

\begin{document}

\maketitle

\begin{abstract}
EFEE 9-13 \\
for next time: \\
3. IE (international express) p38-40) \\
6. Business presentatnions p23-30 \\
\end{abstract}

\section{}
Odcinek - line segment \\
Punkt - point \\
Punkt początowy odcinka - endpoint \\
? Wektor - vector \\

Acute pain \\
\\
Complementary angles -  (suma do 90) \\
Suplementary angles - dopełniające (suma do 180) \\
Angle bisector - półprosta dzieląca kąt na połowy \\
Tangent - tanges \\
The arc (sine, cosine, tangent, cotangent)

\section{Polygons}
\subsection{Triangles}
\subsubsection{Right angled traingle}
Hypotenuse - przeciwprostokątna \\
Legs - przyprostokątne \\
(Wstawić rysunek)
\subsubsection{Scalene triangle}
Each side of different length
\subsection{Quadrilaturals}
Rhombus - romb\\
Trapezium - trapez (one pair of paralel sides) (British English) \\
Trapezoid - trapezoid (no paralel sides) (British English)\\ (in America the other way round)
\subsection{}
Pentagon \\
Hexagon \\
Heptagon\\
Oktagon\\
Nonagon\\
Dekagon\\

\section{circle}
radius \\
circumference \\
diameter \\

Błędne koło - a vicious circle\\
\\
Convex - wypukłe \\
Concave - wklęsłe \\
Concave like cave
\end{document}
