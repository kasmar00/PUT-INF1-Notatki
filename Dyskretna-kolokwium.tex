\documentclass[a4paper,12pt]{article}
\usepackage{polski}
\usepackage{graphicx}
\usepackage[utf8]{inputenc}
\usepackage{geometry}
\newgeometry{tmargin=3cm, bmargin=3cm, lmargin=2cm, rmargin=2cm}
\usepackage{amsmath}
\usepackage{amsfonts}
\usepackage{amssymb}


%opening
\title{Matematyka Dyskretna-Kompendium-Kolokwium 1}
\author{Marcin Kasznia}

\begin{document}

\maketitle
\begin{abstract}
 
\end{abstract}

\section{Logika}
Łączniki logiczne \\
Zdanie odwrotne: $q \to p$ jest zdaniem odwrotnym do $p \to q$ \\
Zdanie przeciwstawne (kontrapozycja): $\neg q \to \neg p$ jest zdaniem przeciwstawnym do $p \to q$ \\
Tautologia: zdanie prawdziwe niezależnie od wartości zdań składowych np. $p\to p$ \\
Zdanie sprzeczne: zdanie fałszywe niezależnie od wartości zdań składowych np. $p \land \neg p $ \\
Zdania logicznie równoważne: zdania które mają takie same wartości niezależnie od wartości zdań składowych ($P \Leftrightarrow Q $) \\
Mówimy że zdanie $P$ implikuje logicznie zdanie $Q$ jeżeli $Q$ ma wartość prawda zawsze wtedy gdy $Q$ ma wartość prawda ($P \Rightarrow Q$)\\
Kwantyfikatory
\subsection{prawa rachunku zdań? (str3 PF 1)}

\section{Teoria mnogości}
Zbiór określamy podając wszystkie elementy tego zbioru lub określając warunek który spełniają wszystkie (i tylko) elementy zbioru (np.: $A=\{x: x \in \mathbb{N} \land x \leq 20 \}$)\\
\textbf{Szczególne zbiory}\\
$\mathbb{P}-$zbiór liczb całkowitych dodatnich $(1, 2, 3,\dots)$ \\
$\mathbb{N}-$zbiór liczb naturualnych $(0, 1, 2, 3,\dots)$ \\
$\mathbb{Z}-$zbiór liczb całkowitych $(\dots, -2, -1, 0, 1, 2,\dots)$ \\
$\mathbb{Q}-$zbiór liczb wymiernych \\
$\mathbb{R}-$zbiór liczb rzeczywistych \\
$\mathbb{C}-$zbiór liczb zespolonych \\
\textbf{Działania na zbiorach}: Suma, Iloczyn (Część wspólna), Różnica, Różnica symetryczna ($A \oplus B=(A\cup B)\setminus (A \cap B)$), Inkluzja (zawieranie) \\
\textbf{Przestrzeń} to zbiór którego każdy zbiór jest podzbiorem \\
\textbf{Dopełnienie zbioru} A do przestrzeni U określane jest jako $\bar A=U \setminus A$ \\
\textbf{Zbiór pusty} jest podzbiorem każdego zbioru $\emptyset$ \\
\textbf{Zbiór potęgowy} zbiór wszystkich podzbiorów danego zbioru-$P(A)$ (moc $|P(A)|=2^{|A|}$)\\
\\
\\
Alfabet-skończony, niepusty zbiór $\Sigma$ (elementami są symbole-litery alfabetu) \\
Słowo-skończony ciąg liter zbioru $\Sigma$ \\
Zbiór $\Sigma ^*$-zbiór wszystich słów zbudowanych z liter alfabetu $\Sigma$ \\
Język nad alfabetem $\Sigma$-dowolny podzbiór zbioru $\Sigma ^*$ 
\subsection{Prawa algebry zbiorów (str5 PF 1)}
\section{}


\end{document}
