\documentclass[a4paper,12pt]{article}
\usepackage{polski}
\usepackage{graphicx}
\usepackage[utf8]{inputenc}
\usepackage{geometry}
\newgeometry{tmargin=3cm, bmargin=3cm, lmargin=2cm, rmargin=2cm}
\usepackage{amsmath}
\usepackage{amsfonts}
\usepackage{amssymb}
\usepackage{textcomp}


%opening
\title{Matematyka Dyskretna-Kompendium-Kolokwium 1}
\author{Marcin Kasznia}

\begin{document}

\maketitle
\begin{abstract}
Na podstawie materiałów prof. Formanowicza i dr Sterny prof. PP\\
W pliku zawarto opracowanie zagadnień poruszanych na ćwiczeniach i obowiązujących na pierwszym kolokwium. W niektórych miejscach (np. prawa rachunku zdań) pominięto rozległe symbole, gdyż większość z nich jest oczywista lub została wyniesiona ze szkoły średniej, w ich miejsce podano odesłania do materiałów od prowadzących.\\
Źródło: https://github.com/kasmar00/PUT-INF1-Notatki/blob/master/Dyskretna-kolokwium.tex \\
\end{abstract}

\section{Logika}
Łączniki logiczne \\
Zdanie odwrotne: $q \to p$ jest zdaniem odwrotnym do $p \to q$ \\
Zdanie przeciwstawne (kontrapozycja): $\neg q \to \neg p$ jest zdaniem przeciwstawnym do $p \to q$ \\
Tautologia: zdanie prawdziwe niezależnie od wartości zdań składowych np. $p\to p$ \\
Zdanie sprzeczne: zdanie fałszywe niezależnie od wartości zdań składowych np. $p \land \neg p $ \\
Zdania logicznie równoważne: zdania które mają takie same wartości niezależnie od wartości zdań składowych ($P \Leftrightarrow Q $) \\
Mówimy że zdanie $P$ implikuje logicznie zdanie $Q$ jeżeli $Q$ ma wartość prawda zawsze wtedy gdy $Q$ ma wartość prawda ($P \Rightarrow Q$)\\
Kwantyfikatory
\subsection{prawa rachunku zdań? (str3 PF 1)}

\section{Teoria mnogości}
Zbiór określamy podając wszystkie elementy tego zbioru lub określając warunek który spełniają wszystkie (i tylko) elementy zbioru (np.: $A=\{x: x \in \mathbb{N} \land x \leq 20 \}$)\\
\textbf{Szczególne zbiory}\\
$\mathbb{P}-$zbiór liczb całkowitych dodatnich $(1, 2, 3,\dots)$ \\
$\mathbb{N}-$zbiór liczb naturualnych $(0, 1, 2, 3,\dots)$ \\
$\mathbb{Z}-$zbiór liczb całkowitych $(\dots, -2, -1, 0, 1, 2,\dots)$ \\
$\mathbb{Q}-$zbiór liczb wymiernych \\
$\mathbb{R}-$zbiór liczb rzeczywistych \\
$\mathbb{C}-$zbiór liczb zespolonych \\
\textbf{Działania na zbiorach}: Suma, Iloczyn (Część wspólna), Różnica, Różnica symetryczna ($A \oplus B=(A\cup B)\setminus (A \cap B)$), Inkluzja (zawieranie) \\
\textbf{Przestrzeń} to zbiór którego każdy zbiór jest podzbiorem \\
\textbf{Dopełnienie zbioru} A do przestrzeni U określane jest jako $\bar A=U \setminus A$ \\
\textbf{Zbiór pusty} jest podzbiorem każdego zbioru $\emptyset$ \\
\textbf{Zbiór potęgowy} zbiór wszystkich podzbiorów danego zbioru-$P(A)$ (moc $|P(A)|=2^{|A|}$)\\
\\
\\
Alfabet-skończony, niepusty zbiór $\Sigma$ (elementami są symbole-litery alfabetu) \\
Słowo-skończony ciąg liter zbioru $\Sigma$ \\
Zbiór $\Sigma ^*$-zbiór wszystich słów zbudowanych z liter alfabetu $\Sigma$ \\
Język nad alfabetem $\Sigma$-dowolny podzbiór zbioru $\Sigma ^*$ 
\subsection{Prawa algebry zbiorów (str5 PF 1)}

\section{Relacje}
\textbf{Para uporządkowana} zbiór  dwuelementowy, gdzie jeden z elemtów wyróżniono \\
\textbf{Iloczyn kartezjański} $[\langle x,y \rangle \in X \times Y] \iff [(x\in X)\wedge (y\in Y)]$ \\
\textbf{Relacją} $\rho$ w iloczynie kartezjańskim $X \times Y$ nazwymamy dowolny podzbiór tego iloczynu \\
\textbf{Dziedzina relacji}: zbiór poprzedników w parach relacji (oznaczamy $D_\rho$) \\
\textbf{Przeciwdziedzina relacji}: zbiór następników w parach relacji (oznaczamy $R_\rho$)
\subsection{Rodzaje relacji}
$\rho \subseteq X \times X$ (Relacja w zbiorze $X$)\\
\textbf{Zwrotna}: $\forall_{x \in X} \langle x, x \rangle \in \rho $ \\
\textbf{Symetryczna}: $\forall_{x \in X} \forall_{y\in X}\langle x,y \rangle \in \rho \implies \langle y, x \rangle \in \rho $ \\
\textbf{Przechodnia}: $\forall_{x \in X} \forall_{y\in X} \forall_{z\in X} [(\langle x, y \rangle \in \rho \wedge \langle y, z \rangle \in \rho ) \implies \langle x, z \rangle \in \rho ]$ \\
\textbf{Antysymetryczna}: $\forall_{x \in X} \forall_{y\in X} [(\langle x, y \rangle \in \rho \wedge \langle y, x \rangle \in \rho ) \implies x=y$ \\
\textbf{Spójna}: $\forall_{x \in X}\forall_{y\in X} \langle x,y \rangle\in\rho \vee \langle y,x\rangle\in\rho$\\
\textbf{Przeciwzwrotna}: $\nexists_{x\in X} \langle x,x\rangle \in \rho$\\
\\
Relacja \textbf{Liniowo porządukjąca zbiór $X$} to relacja jednoczesnie zwrotna, przechodnia, antysymetryczna i spójna \\
\textbf{Relacja równoważności}: jednoczesnie zwrotna, symetryczna i przechodnia (oznaczamy $ \sim $)

\section{Funkcje}
Relację $\rho \subseteq X \times Y$ spełniającą warunki: $\forall_{x\in X} \exists_{y \in Y} \langle x, y \rangle \in \rho $ oraz $\forall_{x\in X} \forall_{y \in Y} \forall_{z \in Y} \langle x,y \rangle \in \rho \wedge\langle x, z \rangle \in \rho \implies y=z$ nazywamy funkcją

\subsection{UZUPEŁNIĆ!}

\section{Kombinatoryka}
\textbf{Prawo sumy}: $|A \cup B|=|A|+|B|-|A \cap B|$ (liczba sposobów na jakie można wybrać element należący do jednego z dwu zbiorów) \\
\textbf{Prawo iloczynu}: $|A\times B|=|A|*|B|$ liczba sposobów na jakie można wybrać uporządkowaną parę elementów jest równa liczbie możliwości na jakie można wybrac pierwszy element przemnożonej przez liczbe możliwości na jakie można wybrac drugi element 
\subsection{Wariacja z powtórzeniami} $\bar V (n, k)=n^k$ \\
Liczba sposobów na jakie można utworzyć $k$ wyrazowy ciąg z $n$ elementowego zbioru (wyrazy mogą sie powtarzać) \\
Liczba możliwych rozmieszczeń $k$ rozróżnialnych elementów w $n$ rozróżnialnych pudełkach 
\subsection{Wariacja bez powtórzeń} $V(n,k)=\frac{n!}{n-k!}$ \\
Liczba sposobów na jakie można utworzyć $k$ elementowy ciąg z $n$ elementowego zbioru (wyarzy nie możą się powtarzać)
\subsection{Permutacja bez powtórzeń} $P(n,k)=V(n,k)=\frac{n!}{n-k!}$\\
Liniowe uporządkowanie $k$ rozróżnialnych elementów ze zbioru $n$ elementowego \\
Permutacja pusta ($k=0$) $P(n,0)=1$ \\
Permutacja elementów zbioru ($n=k$) $P(n, n)=n!$
\subsection{Permutacja z powtórzeniami} $P(n,n_1,n_2,\dots ,n_k)=\frac{n!}{n_1!*n_2!*\dots *n_k!}= {{n}\choose{n_1,n_2,\dots, n_k}}$ \\
Permutacją $n$ elementową z powtórzeniami zbioru  $A=a_1,a_2,\dots,a_k$, w której element $a_1$ powtarza się $n_1$ razy, $\dots$, element $a_k$ powtarza się $n_k$ razy, ($n_1+ \dots + n_k= n$), nazywamy każdy ciąg $n$ wyrazowy, w którym poszczególne elementy zbioru $A$ powtarzają się wskazaną liczbę razy.\\

$P(n,n_1,n_2,\dots ,n_k)$ to liczba: 
\begin{itemize} \itemsep1pt \parskip0pt \parsep0pt
 \item $n$ elementowych permutacji z powtórzeniami elementów $k$ typów
 \item rozmieszczeń $n$ rozróżnialnych obiektów w $k$ rozróżnialnych pudełkach, takich że w $i$-tym pudełku znajduje się $n_i$ obiektów,
 \item podziałów uporządkowanych zbioru.
\end{itemize}
\subsection{Kombinacja bez powtórzeń} $C(n,k)=\frac{n!}{(n-k)!*k!}={{n}\choose{k}}$ \\
$k$ elementowy podzbiór zbioru $n$ elementowego 
\subsection{Kombinacja z powtórzeniami} $\bar C(n,k)={{n+k-1}\choose{k}}=\frac{(n+k-1)!}{k!(n-1)!}$ \\
Zestaw $k$ elementów, z których każdy należy do jednego z $n$ rodzajów elementów
Liczba $\bar C(n,k)$ określa liczbę:
\begin{itemize} \itemsep1pt \parskip0pt \parsep0pt
 \item $k$ elementowych kombinacji z powtórzeniami ze zbioru $n$ elementowego
 \item rozmieszczeń $k$ identycznych elementów w $n$ rozróżnialnych pudełkach
 \item całkowitoliczbowych rozwiązań równania postaci $x_1+x_2+\dots +x_n=k$ gdzie $x_i>0$
\end{itemize}

\subsection{Tabelki}
\begin{tabular}{c|c|c|c}
uporządkowanie & powtórzenia & obiekt & liczba\\
\hline 
 + & $-$ & wariacja bez powtórzeń & $V(n,k)=\frac{n!}{(n-k)!}$ \\
 + & + & wariacja z powtórzeniami & $\bar V(n,k)=n^k$ \\
 $-$ & $-$ & kombinacja bez powtórzeń & $C(n,k)={{n}\choose{k}}$ \\
 $-$ & + & kombinacja z powtórzeniami & $\bar C(n,k)={{n+k-1}\choose{k}}$ \\
\end{tabular} \\
\begin{tabular}{c|c|c|c}
 k elementów & n pudełek & obiekt & liczba \\
 \hline
 identyczne & rozróżnialne & kombinacja z powtórzeniami & $\overline{C}(n,k)={{n+k-1}\choose{k}}$\\
 rozróżnialne & rozróżnialne & wariacja z powtórzeniami & $\bar V(n,k)=n^k$ \\
 \hline
 rozróżnialne & identyczne & liczby Stirlinga 2-giego rodzaju & ? \\
 identyczne & identyczne & podziały liczb całkowitych & ?\\
\end{tabular}

\section{Indukcja}

\textbf{Współczynnik dwumianowy}: symbol Newtona ${{n}\choose {k}}=\frac{n!}{k!(n-k)!}$ \\
\textbf{Wzór na n-tą potęgę dwumianu}: $(x+y)^n=\sum_{k=0}^n {{n}\choose{k}}x^ky^{n-k}$ \\
${{n}\choose{k}}={{n}\choose{n-k}}$ (reguła symetrii), 
${{n}\choose{k}}={{n-1}\choose{k}}+{{n-1}\choose{k-1}}$ (reguła dodawania), 
$k{{n}\choose{k}}=n{{n-1}\choose{k-1}}$ (reguła pochłaniania), 
${{m+n}\choose{k}}=\sum_{s=0}^k {{n}\choose{s}}{{n}\choose{k-s}}$ (tożsamość Cauchy'ego) \\
\textbf{Trójkąt Pascala}: metoda obliczania wartości współczynnika Newtona {reguła dodawania}\\
\textbf{Współczynnik wielomianowy}: ${{n}\choose{k_1, k_2, \dots ,k_r}}={{n}\choose{k_1}}{{n-k_1}\choose{k_2}}{{n-k_1-k_2}\choose{k_3}}\dots {{n-k_1-\hdots -k_{r-1}}\choose{k_r}}=\frac{n!}{k!k_2!\dots k_r!}$, występuje w rozwinięciu wielomianu $(x_1+x_2+\dots +x_r)^n=\sum {{n}\choose{k_1, k_2, \dots ,k_r}}x_1^{k_1}x_2^{k_2}\dots x_ r^{k_r}$
\subsection{Pierwsza zasada indukcji matematycznej}
Inaczej \textbf{Zasada skończonej indukcji matematycznej} \\
Jeśli $S(1)$ jest prawdziwe i dla dowolnego $k \in \mathbb{P}$, $S(k+1)$ jest prawdziwe, jeśli $S(k)$ jest prawdziwe, to $S(n)$ jest prawdziwe dla wszystkich $n \in \mathbb{P}$ \\ %formalny zapis u Sterny na wykładzie

\textbf{Dowód indukcyjny}
\begin{itemize} \itemsep1pt \parskip0pt \parsep0pt
 \item Sformułowanie hipotezy indukcyjnej $S(k)$
 \item Dowiedzenie prawdziwości warunku początkowego $S(n_0)$
 \item dowiedzenia prawdziwości kroku indukcyjnego (prawdziwości implikacji $S(k)\rightarrow S(k+1)$)
\end{itemize}
\subsection{Druga zasada indukcji matematycznej}
Inaczej \textbf{Zasada silnej indukcji matematycznej} \\
Jeśli $S(n_0), S(n_0+1), \dots ,S(n_1-1), S(n_1)$ są prawdziwe i dla dowolnego $k\in\mathbb{P}, k\geq n_1$ $S(k+1)$ jest prawdziwe, jeśli $S(n_0), S(n_0+1), \dots ,S(k-1), S(k)$ są prawdziwe to $S(n)$ jest prawdziwe dla wszystich $n \in\mathbb{P}, n\geq n_0$. (przy założeniu że: $n_0, n_1 \in\mathbb{P}$, $n_0\leq n_1$) \\
Zasadę tą stosuje się:
\begin{itemize}\itemsep1pt \parskip0pt \parsep0pt
 \item jeśli prawdziwość analizowanego zdania wynika z prawdziwości pewnych zdań poprzedzających
 \item jeśli krok indukcyjny nie jest prawdziwy dla pewnych początkowych wartości $k$
 \item w analizie definicji rekurencyjnych, w których pewne wyrazy określone są za pomocą wyrazów innych niż bezpośrednio poprzedzające
\end{itemize}

\section{Rekurencja}
Ciąg jest zdefiniowany rekurencyjnie jeśli: określony jest skońcozny zbiór wyrazów ciągu (warunek początkowy), pozostałe wyrazy zdefiniowane są za pomocą poprzednich wyrazów (warunek rekurencyjny). Poprawność definicji dowodzi się w oparciu o indukcję matematyczną.
\subsection{Wzory jawne}
Nie są znane metody ogólne rozwiązywania zależności rekurencyjnych niejednorodnych.
\subsubsection{Zależność pierwszego rzędu}
$a_{n+1}+ca_n=f(n)$ \\
Gdzie: $n\in\mathbb{N}$, $c=const$, $f(n)$-pewna funkcja określona na zbiorze $\mathbb{N}$\\
Gdy $\forall_{n \in\mathbb{N}}f(n)=0$ to zależność nazywamy jednorodną ($a_{n+1}=da_n$) (np. $a_{n+1}=3a_n,  a_0=5, n\geq0$) \\
Dla pozostałych funkcji jest to zależność niejednorodna (np. $a_n=a_{n-1}+(n-1), a_1=0, n\geq2$).
\textbf{Liniowe jednorodne zależności 1-szego rzędu}: \\
Wzór rekurencyjny: $a_{n_+1}=da_n$ dla $n\geq0, d=const, a_0=const$ \\
Wzór jawny: $a_n=a_0d^n$
\subsubsection{Zależność drugiego rzędu}
$c_na_n+c_{n-1}a_{n-1}+c_{n-2}a_{n-2}=f(n)$\\
Gdzie: $n\in\mathbb{N}, n\geq2$, $c_n,c_{n-1}, c_{n-2}\in\mathbb{R}, c_n\neq0, c_{n-2}\neq0$, $f(n)$-pewna funkcja określona na zbiorze liczb naturalnych.\\
Gdy $\forall_{n \in\mathbb{N}}f(n)=0$ to zależność nazywamy jednorodną, dla pozostałych funkcji  jest to zależność niejednorodna \\
\textbf{Liniowe jednorodne zależności 2-giego rzędu}\\
Przewidywane rozwiązanie zależnosći $c_na_n+c_{n-1}a_{n-1}+c_{n-2}a_{n-2}=0$ ma postać $a_n=cr^n$, gdzie $c\neq0, r\neq0$. Pod podstawieniu do zależnosci rekurencyjnej otrzymujemy równanie $c_ncr^n+c_{n-1}cr^{n-1}+c_{n-2}cr^{n-2}=0$ które upraszcza się do \textbf{równania charakterystyczego} postaci: $c_nr^2+c_{n-1}r+c_{n-2}=0$.\\
Jeśli równanie to, ma dwa różne pierwiastki $r_1, r_2$ to: $a_n=\hat c_1r_1^n+\hat c_2r_2^n$, jeden pierwiastek podwójny $r$ to: $a_n=\hat c_1r^n+\hat c_2nr^n$ (stałe $\hat c_1, \hat c_2$ wyznaczamy z wartości początkowych ciągu).
\subsubsection{Zależnosć k-tego rzędu}
$c_na_n+c_{n-1}a_{n-1}+\dots + c_{n-k}a_{n-k}=f(n)$\\
Gdzie: $n,k\in\mathbb{N}, n\geq k$, $c_n,c_{n-1}, c_k\in\mathbb{R}, c_n\neq0, c_{n-k}\neq0$, $f(n)$-pewna funkcja określona na zbiorze liczb naturalnych.\\
Gdy $\forall_{n \in\mathbb{N}}f(n)=0$ to zależność nazywamy jednorodną, dla pozostałych funkcji  jest to zależność niejednorodna. \\
\textbf{Liniowe jednorodne zależności k-tego rzędu}\\
Uogólnienie z drugiego rzędu, szczegóły u Sterny na prezentacji
\subsection{Problemy rekurencyjne}
\textbf{Wieże z Hanoi}: w jakiej minimalnej liczbie ruchów, $T_n$, można przenieść wszystkie krązki z jednego pręta na drugi, przenosząc je pojedynczo i nigdy nie kładąc większego na mniejszy?\\
$T_n=2T_{n-1}+1, T_0=0$ (zależność niejednorodna, wzór jawny: $T_n=2^n-1$)
\section{Liczby Szczególne}


\end{document}
