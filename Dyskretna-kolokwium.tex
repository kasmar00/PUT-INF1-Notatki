\documentclass[a4paper,12pt]{article}
\usepackage{polski}
\usepackage{graphicx}
\usepackage[utf8]{inputenc}
\usepackage{geometry}
\newgeometry{tmargin=3cm, bmargin=3cm, lmargin=2cm, rmargin=2cm}
\usepackage{amsmath}
\usepackage{amsfonts}
\usepackage{amssymb}
\usepackage{textcomp}
\usepackage{enumerate}
\usepackage{bigfoot}

\DeclareRobustCommand{\Stirling}{\genfrac \{ \} {0pt}{}} %liczby stirlinga 2-giego rodzaju
\DeclareRobustCommand{\stirling}{\genfrac [ ] {0pt}{}} %liczby stirlinga 1-szego rodzaju


%opening
\title{Matematyka Dyskretna-Notatka}
\author{Marcin Kasznia}

\begin{document}

\maketitle
\begin{abstract}
Na podstawie materiałów prof. Formanowicza i dr Sterny prof. PP\\
W pliku zawarto opracowanie zagadnień poruszanych na ćwiczeniach i obowiązujących na pierwszym kolokwium. W niektórych miejscach (np. prawa rachunku zdań) pominięto rozległe symbole, gdyż większość z nich jest oczywista lub została wyniesiona ze szkoły średniej, w ich miejsce podano odesłania do materiałów od prowadzących.\\
Źródło: https://github.com/kasmar00/PUT-INF1-Notatki/blob/master/Dyskretna-kolokwium.tex \\
\end{abstract}

\section{Logika}
Łączniki logiczne \\
\textbf{Zdanie odwrotne}: $q \to p$ jest zdaniem odwrotnym do $p \to q$ \\
\textbf{Zdanie przeciwstawne} (kontrapozycja): $\neg q \to \neg p$ jest zdaniem przeciwstawnym do $p \to q$ \\
\textbf{Tautologia}: zdanie prawdziwe niezależnie od wartości zdań składowych np. $p\to p$ \\
\textbf{Zdanie sprzeczne}: zdanie fałszywe niezależnie od wartości zdań składowych np. $p \land \neg p $ \\
\textbf{Zdania logicznie równoważne}: zdania które mają takie same wartości niezależnie od wartości zdań składowych ($P \Leftrightarrow Q $) \\
Mówimy że zdanie $P$ \textbf{implikuje logicznie} zdanie $Q$ jeżeli $Q$ ma wartość prawda zawsze wtedy gdy $P$ ma wartość prawda ($P \Rightarrow Q$)\\
\textbf{Kwantyfikatory}\\
Reguła \textbf{podstawienia}: Jeśli zdanie złożone $P$ jest tautologią i wszystkie wystąpienia pewnej zmiennej $p$ występującej w zdaniu $P$ zastąpimy zdaniem $R$ to otrzymane zdanie złożone $P_1$ będzie również tautologią. Jeżeli zdanie złożone $P$ zawiera zdanie $Q$, $R$ jest zdaniem takim, że $Q\Leftrightarrow R$ i jeśli w P zastąpimy jedno lub więcej wystąpień $Q$ przez $R$ to otrzymamy zdanie złożone $P_2$ takie, że $P \Leftrightarrow P_2$
\subsection{prawa rachunku zdań (str3 PF 1)}

\section{Teoria mnogości}
\textbf{Pojęcia pierwotne}: zbioru oraz należenia do zbioru\\
Zbiór określamy podając wszystkie elementy tego zbioru lub określając warunek który spełniają wszystkie (i tylko) elementy zbioru (np.: $A=\{x: x \in \mathbb{N} \land x \leq 20 \}$)\\
\textbf{Szczególne zbiory}\\
$\mathbb{P}-$zbiór liczb całkowitych dodatnich $(1, 2, 3,\dots)$ \\
$\mathbb{N}-$zbiór liczb naturualnych $(0, 1, 2, 3,\dots)$ \\
$\mathbb{Z}-$zbiór liczb całkowitych $(\dots, -2, -1, 0, 1, 2,\dots)$ \\
$\mathbb{Q}-$zbiór liczb wymiernych \\
$\mathbb{R}-$zbiór liczb rzeczywistych \\
$\mathbb{C}-$zbiór liczb zespolonych \\
\textbf{Działania na zbiorach}: Suma, Iloczyn (Część wspólna), Różnica, Różnica symetryczna ($A \oplus B=(A\cup B)\setminus (A \cap B)$)\\
\textbf{Inkluzja (zawieranie)}: $A\subseteq B \Leftrightarrow [(a \in A)\Rightarrow (a\in B)]$ ($A$ nazywamy podzbiorem $B$, $B$ nazywamy nadzbiorem $A$) \\
\textbf{Podzbiór właściwy} jeżeli $A\subseteq B \wedge A\neq B$ $A$ to podzbiór właściwy ($A\subset B$)\\
\textbf{Przestrzeń} to zbiór którego każdy zbiór jest podzbiorem \\
\textbf{Dopełnienie zbioru} A do przestrzeni U określane jest jako $\bar A=U \setminus A$ \\
\textbf{Zbiór pusty} jest podzbiorem każdego zbioru $\emptyset$ \\
\textbf{Zbiór potęgowy} zbiór wszystkich podzbiorów danego zbioru-$P(A)$ (moc $|P(A)|=2^{|A|}$)\\
\\
\\
Alfabet-skończony, niepusty zbiór $\Sigma$ (elementami są symbole-litery alfabetu) \\
Słowo-skończony ciąg liter zbioru $\Sigma$ \\
Zbiór $\Sigma ^*$-zbiór wszystich słów zbudowanych z liter alfabetu $\Sigma$ \\
Język nad alfabetem $\Sigma$-dowolny podzbiór zbioru $\Sigma ^*$ 
\subsection{Prawa algebry zbiorów (str5 PF 1)}

\section{Relacje}
\textbf{Para uporządkowana} zbiór  dwuelementowy, gdzie jeden z elemtów wyróżniono \\
\textbf{Iloczyn kartezjański} $[\langle x,y \rangle \in X \times Y] \iff [(x\in X)\wedge (y\in Y)]$ \\
\textbf{Relacją} $\rho$ w iloczynie kartezjańskim $X \times Y$ nazwymamy dowolny podzbiór tego iloczynu \\
\textbf{Dziedzina relacji}: zbiór poprzedników w parach relacji (oznaczamy $D_\rho$) \\
\textbf{Przeciwdziedzina relacji}: zbiór następników w parach relacji (oznaczamy $R_\rho$)
\subsection{Rodzaje relacji}
$\rho \subseteq X \times X$ (Relacja w zbiorze $X$)\\
\textbf{Zwrotna}: $\forall_{x \in X} \langle x, x \rangle \in \rho $ \\
\textbf{Symetryczna}: $\forall_{x \in X} \forall_{y\in X}\langle x,y \rangle \in \rho \implies \langle y, x \rangle \in \rho $ \\
\textbf{Przechodnia}: $\forall_{x \in X} \forall_{y\in X} \forall_{z\in X} [(\langle x, y \rangle \in \rho \wedge \langle y, z \rangle \in \rho ) \implies \langle x, z \rangle \in \rho ]$ \\
\textbf{Antysymetryczna}: $\forall_{x \in X} \forall_{y\in X} [(\langle x, y \rangle \in \rho \wedge \langle y, x \rangle \in \rho ) \implies x=y]$ \\
\textbf{Spójna}: $\forall_{x \in X}\forall_{y\in X} \langle x,y \rangle\in\rho \vee \langle y,x\rangle\in\rho$\\
\textbf{Przeciwzwrotna}: $\nexists_{x\in X} \langle x,x\rangle \in \rho$\\
\\
Relacja \textbf{Liniowo porządukjąca zbiór $X$} to relacja jednoczesnie zwrotna, przechodnia, antysymetryczna i spójna \\
\textbf{Relacja równoważności}: jednoczesnie zwrotna, symetryczna i przechodnia (oznaczamy $ \sim $)

\section{Funkcje}
Relację $\rho \subseteq X \times Y$ spełniającą warunki: $\forall_{x\in X} \exists_{y \in Y} \langle x, y \rangle \in \rho $ oraz $\forall_{x\in X} \forall_{y \in Y} \forall_{z \in Y} \langle x,y \rangle \in \rho \wedge\langle x, z \rangle \in \rho \Rightarrow y=z$ nazywamy funkcją.\\
Zbiór $X$ nazywamy dziedziną funkcji $f$ i oznaczamy przez $D_f$. Każdy element $x \in X$ nazywamy argumentem funkcji $f$, natomiast $f(x)\in Y$ nazwyamy wartością funkcji $f$ dla argumentu $x$. \\

Niech $f:X\rightarrow Y$ oraz $A \subseteq X$.\\
\textbf{Obraz} zbioru $A$: zbiór wartości jakie przyjmuje funkcja $f$ na zbiorze $A$ oznaczamy przez $f(A)$. $$f(A)=\{ f(x)\in Y:x\in A \}$$
\textbf{Przeciwdziedzina} funkcji $f$: Zbiór $f(X)$ będący podzbiorem zbioru $Y$ oznaczamy $R_f$. $$R_f=\{f(x)\in Y: x\in D_f\}$$
Funkcja \textbf{różnowartościowa} (na zbiorze $A$) $\forall_{x_1,x_2\in A} x_1\neq x_2  \Rightarrow f(x_1)\neq f(x_2)$. Jeżeli $A=X$ to $f$ jest funkcją różnowartościową.\\
Jeżeli $R_f=Y$, tzn. jeżeli każdy element zbioru $Y$ jest przyporządkowany co najmniej jednemu elementowi zbioru $X$ to mówimy, że funkcja $f$ przekształca (odwzorowuje) zbiór $X$ \textbf{na} zbiór $Y$.\\
\textbf{Przekształcenie wzajemnie równoznaczne}: funkcja która jest jednocześnie \textbf{różnowartościowa} i odwzorowuje \textbf{na}.\\
\textbf{Funkcja odwrotna} (tylko dla funkcji które są przekształceniem wzajemnie jednoznacznym) do $f$ to taka $f^{-1}$, która każdemu $y\in Y$ przyporządkowuje $x \in X$ taki, że $f(x)=y$. Zatem $\forall_{y\in Y} f(f^{-1}(y))=y$ \\
\textbf{Funkcje złożone} (tu chyba nie trzeba dużo pisać, może dopisze później) \\
\textbf{Funkcja podłoga} $\forall_{x\in \mathbb{R}} \lfloor x \rfloor=y\in \mathbb{Z}$ takie że $\{ y\leq x \wedge y+1>x \}$ \\
\textbf{Funkcja sufit} $\forall_{x\in \mathbb{R}} \lceil x \rceil=y\in \mathbb{Z}$ takie że $\{ y\geq x \wedge y-1<x \}$
\subsection{Rzędzy wielkosci funkcji}
\textbf{Notacja $O$} $f$ jest co najwyżej rzędu $g$ (ogranicza z góry)\\
$O(g(N))=\{f(n):$ istnieją dodatnie stałe $c$ i $n_0$ takie, że $0\leq f(n) \leq cg(n)$ dla wszystkich $n \geq n_0 \}$\\
\textbf{Notacja $o$} $f$ jest niższego rzędu niż $g$ \\
\textbf{Notacja $\Omega$} $f$ jest niższego rzędu niż $g$\\
\textbf{Notacja $\omega$} $f$ jest wyższego rzędu niż $g$ \\
\textbf{Notacja $\Theta$} $f$ jest dokładnie rzędu $g$\\
\verb+https://pl.wikipedia.org/wiki/Asymptotyczne_tempo_wzrostu+\\

\section{Kombinatoryka}
\textbf{Prawo sumy}: $|A \cup B|=|A|+|B|-|A \cap B|$ (liczba sposobów na jakie można wybrać element należący do jednego z dwu zbiorów) \\
\textbf{Prawo iloczynu}: $|A\times B|=|A|*|B|$ liczba sposobów na jakie można wybrać uporządkowaną parę elementów jest równa liczbie możliwości na jakie można wybrac pierwszy element przemnożonej przez liczbe możliwości na jakie można wybrac drugi element 
\subsection{Wariacja z powtórzeniami} $\bar V (n, k)=n^k$ \\
Liczba sposobów na jakie można utworzyć $k$ wyrazowy ciąg z $n$ elementowego zbioru (wyrazy mogą sie powtarzać) \\
Liczba możliwych rozmieszczeń $k$ rozróżnialnych elementów w $n$ rozróżnialnych pudełkach 
\subsection{Wariacja bez powtórzeń} $V(n,k)=\frac{n!}{(n-k)!}$ \\
Liczba sposobów na jakie można utworzyć $k$ elementowy ciąg z $n$ elementowego zbioru (wyarzy nie możą się powtarzać)
\subsection{Permutacja bez powtórzeń} $P(n,k)=V(n,k)=\frac{n!}{(n-k)!}$\\
Liniowe uporządkowanie $k$ rozróżnialnych elementów ze zbioru $n$ elementowego \\
Permutacja pusta ($k=0$) $P(n,0)=1$ \\
Permutacja elementów zbioru ($n=k$) $P(n, n)=n!$
\subsection{Permutacja z powtórzeniami} $P(n,n_1,n_2,\dots ,n_k)=\frac{n!}{n_1!*n_2!*\dots *n_k!}= {{n}\choose{n_1,n_2,\dots, n_k}}$ \\
Permutacją $n$ elementową z powtórzeniami zbioru  $A=a_1,a_2,\dots,a_k$, w której element $a_1$ powtarza się $n_1$ razy, $\dots$, element $a_k$ powtarza się $n_k$ razy, ($n_1+ \dots + n_k= n$), nazywamy każdy ciąg $n$ wyrazowy, w którym poszczególne elementy zbioru $A$ powtarzają się wskazaną liczbę razy.\\

$P(n,n_1,n_2,\dots ,n_k)$ to liczba: 
\begin{itemize} \itemsep1pt \parskip0pt \parsep0pt
 \item $n$ elementowych permutacji z powtórzeniami elementów $k$ typów
 \item rozmieszczeń $n$ rozróżnialnych obiektów w $k$ rozróżnialnych pudełkach, takich że w $i$-tym pudełku znajduje się $n_i$ obiektów,
 \item podziałów uporządkowanych zbioru.
\end{itemize}
\subsection{Kombinacja bez powtórzeń} $C(n,k)=\frac{n!}{(n-k)!*k!}={{n}\choose{k}}$ \\
$k$ elementowy podzbiór zbioru $n$ elementowego 
\subsection{Kombinacja z powtórzeniami} $\bar C(n,k)={{n+k-1}\choose{k}}=\frac{(n+k-1)!}{k!(n-1)!}$ \\
Zestaw $k$ elementów, z których każdy należy do jednego z $n$ rodzajów elementów
Liczba $\bar C(n,k)$ określa liczbę:
\begin{itemize} \itemsep1pt \parskip0pt \parsep0pt
 \item $k$ elementowych kombinacji z powtórzeniami ze zbioru $n$ elementowego
 \item rozmieszczeń $k$ identycznych elementów w $n$ rozróżnialnych pudełkach
 \item całkowitoliczbowych rozwiązań równania postaci $x_1+x_2+\dots +x_n=k$ gdzie $x_i>0$
\end{itemize}

\subsection{Tabelki}
\begin{tabular}{c|c|c|c}
uporządkowanie & powtórzenia & obiekt & liczba\\
\hline 
 \checkmark & $-$ & wariacja bez powtórzeń & $V(n,k)=\frac{n!}{(n-k)!}$ \\
 + & + & wariacja z powtórzeniami & $\bar V(n,k)=n^k$ \\
 $-$ & $-$ & kombinacja bez powtórzeń & $C(n,k)={{n}\choose{k}}$ \\
 $-$ & + & kombinacja z powtórzeniami & $\bar C(n,k)={{n+k-1}\choose{k}}$ \\
\end{tabular} \\
\begin{tabular}{c|c|c|c}
 k elementów & n pudełek & obiekt & liczba \\
 \hline
 identyczne & rozróżnialne & kombinacja z powtórzeniami & $\overline{C}(n,k)={{n+k-1}\choose{k}}$\\
 rozróżnialne & rozróżnialne & wariacja z powtórzeniami & $\bar V(n,k)=n^k$ \\
 \hline
 rozróżnialne & identyczne & liczby Stirlinga 2-giego rodzaju & ? \\
 identyczne & identyczne & podziały liczb całkowitych & ?\\
\end{tabular}

\section{Indukcja}

\textbf{Współczynnik dwumianowy}: symbol Newtona ${{n}\choose {k}}=\frac{n!}{k!(n-k)!}$ \\
\textbf{Wzór na n-tą potęgę dwumianu}: $(x+y)^n=\sum_{k=0}^n {{n}\choose{k}}x^ky^{n-k}$ \\
${{n}\choose{k}}={{n}\choose{n-k}}$ (reguła symetrii), 
${{n}\choose{k}}={{n-1}\choose{k}}+{{n-1}\choose{k-1}}$ (reguła dodawania), 
$k{{n}\choose{k}}=n{{n-1}\choose{k-1}}$ (reguła pochłaniania), 
${{m+n}\choose{k}}=\sum_{s=0}^k {{n}\choose{s}}{{n}\choose{k-s}}$ (tożsamość Cauchy'ego) \\
\textbf{Trójkąt Pascala}: metoda obliczania wartości współczynnika Newtona (reguła dodawania)\\
\textbf{Współczynnik wielomianowy}: ${{n}\choose{k_1, k_2, \dots ,k_r}}={{n}\choose{k_1}}{{n-k_1}\choose{k_2}}{{n-k_1-k_2}\choose{k_3}}\dots {{n-k_1-\hdots -k_{r-1}}\choose{k_r}}=\frac{n!}{k_1!k_2!\dots k_r!}$, występuje w rozwinięciu wielomianu $(x_1+x_2+\dots +x_r)^n=\sum {{n}\choose{k_1, k_2, \dots ,k_r}}x_1^{k_1}x_2^{k_2}\dots x_ r^{k_r}$\\
\textbf{Zasada dobrego uporządkowania}: każdy niepiusty zbiór jest dobrze uporządkowany wtw. gdy zawiera on element najmniejszy
\subsection{Pierwsza zasada indukcji matematycznej}
Inaczej \textbf{Zasada skończonej indukcji matematycznej} \\
Jeśli $S(1)$ jest prawdziwe i dla dowolnego $k \in \mathbb{P}$, $S(k+1)$ jest prawdziwe, jeśli $S(k)$ jest prawdziwe, to $S(n)$ jest prawdziwe dla wszystkich $n \in \mathbb{P}$ \\ %formalny zapis u Sterny na wykładzie

\textbf{Dowód indukcyjny}
\begin{itemize} \itemsep1pt \parskip0pt \parsep0pt
 \item Sformułowanie hipotezy indukcyjnej $S(k)$
 \item Dowiedzenie prawdziwości warunku początkowego $S(n_0)$
 \item dowiedzenia prawdziwości kroku indukcyjnego (prawdziwości implikacji $S(k)\rightarrow S(k+1)$)
\end{itemize}
\subsection{Druga zasada indukcji matematycznej}
Inaczej \textbf{Zasada silnej indukcji matematycznej} \\
Jeśli $S(n_0), S(n_0+1), \dots ,S(n_1-1), S(n_1)$ są prawdziwe i dla dowolnego $k\in\mathbb{P}, k\geq n_1$ $S(k+1)$ jest prawdziwe, jeśli $S(n_0), S(n_0+1), \dots ,S(k-1), S(k)$ są prawdziwe to $S(n)$ jest prawdziwe dla wszystich $n \in\mathbb{P}, n\geq n_0$. (przy założeniu że: $n_0, n_1 \in\mathbb{P}$, $n_0\leq n_1$) \\
Zasadę tą stosuje się:
\begin{itemize}\itemsep1pt \parskip0pt \parsep0pt
 \item jeśli prawdziwość analizowanego zdania wynika z prawdziwości pewnych zdań poprzedzających
 \item jeśli krok indukcyjny nie jest prawdziwy dla pewnych początkowych wartości $k$
 \item w analizie definicji rekurencyjnych, w których pewne wyrazy określone są za pomocą wyrazów innych niż bezpośrednio poprzedzające
\end{itemize}
\subsection{Niezmienniki pętli ?}

\section{Rekurencja}
Ciąg jest zdefiniowany rekurencyjnie jeśli: określony jest skońcozny zbiór wyrazów ciągu (warunek początkowy), pozostałe wyrazy zdefiniowane są za pomocą poprzednich wyrazów (warunek rekurencyjny). Poprawność definicji dowodzi się w oparciu o indukcję matematyczną.
\subsection{Wzory jawne}
Nie są znane metody ogólne rozwiązywania zależności rekurencyjnych niejednorodnych.
\subsubsection{Zależność pierwszego rzędu}
$a_{n+1}+ca_n=f(n)$ \\
Gdzie: $n\in\mathbb{N}$, $c=const$, $f(n)$-pewna funkcja określona na zbiorze $\mathbb{N}$\\
Gdy $\forall_{n \in\mathbb{N}}f(n)=0$ to zależność nazywamy jednorodną ($a_{n+1}=da_n$) (np. $a_{n+1}=3a_n,  a_0=5, n\geq0$) \\
Dla pozostałych funkcji jest to zależność niejednorodna (np. $a_n=a_{n-1}+(n-1), a_1=0, n\geq2$).
\textbf{Liniowe jednorodne zależności 1-szego rzędu}: \\
Wzór rekurencyjny: $a_{n_+1}=da_n$ dla $n\geq0, d=const, a_0=const$ \\
Wzór jawny: $a_n=a_0d^n$
\subsubsection{Zależność drugiego rzędu}
$c_na_n+c_{n-1}a_{n-1}+c_{n-2}a_{n-2}=f(n)$\\
Gdzie: $n\in\mathbb{N}, n\geq2$, $c_n,c_{n-1}, c_{n-2}\in\mathbb{R}, c_n\neq0, c_{n-2}\neq0$, $f(n)$-pewna funkcja określona na zbiorze liczb naturalnych.\\
Gdy $\forall_{n \in\mathbb{N}}f(n)=0$ to zależność nazywamy jednorodną, dla pozostałych funkcji  jest to zależność niejednorodna \\
\textbf{Liniowe jednorodne zależności 2-giego rzędu}\\
Przewidywane rozwiązanie zależnosći $c_na_n+c_{n-1}a_{n-1}+c_{n-2}a_{n-2}=0$ ma postać $a_n=cr^n$, gdzie $c\neq0, r\neq0$. Pod podstawieniu do zależnosci rekurencyjnej otrzymujemy równanie $c_ncr^n+c_{n-1}cr^{n-1}+c_{n-2}cr^{n-2}=0$ które upraszcza się do \textbf{równania charakterystyczego} postaci: $c_nr^2+c_{n-1}r+c_{n-2}=0$.\\
Jeśli równanie to, ma dwa różne pierwiastki $r_1, r_2$ to: $a_n=\hat c_1r_1^n+\hat c_2r_2^n$, jeden pierwiastek podwójny $r$ to: $a_n=\hat c_1r^n+\hat c_2nr^n$ (stałe $\hat c_1, \hat c_2$ wyznaczamy z wartości początkowych ciągu).
\subsubsection{Zależnosć k-tego rzędu}
$c_na_n+c_{n-1}a_{n-1}+\dots + c_{n-k}a_{n-k}=f(n)$\\
Gdzie: $n,k\in\mathbb{N}, n\geq k$, $c_n,c_{n-1}, c_k\in\mathbb{R}, c_n\neq0, c_{n-k}\neq0$, $f(n)$-pewna funkcja określona na zbiorze liczb naturalnych.\\
Gdy $\forall_{n \in\mathbb{N}}f(n)=0$ to zależność nazywamy jednorodną, dla pozostałych funkcji  jest to zależność niejednorodna. \\
\textbf{Liniowe jednorodne zależności k-tego rzędu}\\
Uogólnienie z drugiego rzędu, szczegóły u Sterny na prezentacji
\subsection{Problemy rekurencyjne}
\textbf{Wieże z Hanoi}: w jakiej minimalnej liczbie ruchów, $T_n$, można przenieść wszystkie krązki z jednego pręta na drugi, przenosząc je pojedynczo i nigdy nie kładąc większego na mniejszy?\\
$T_n=2T_{n-1}+1, T_0=0$ (zależność niejednorodna, wzór jawny: $T_n=2^n-1$)

\section{Liczby Szczególne}
%na egzamin liczby eulera, euklidesa, mersennea
\subsection{Liczby Stirlinga drugiego rodzaju}
$S(n,k)=\Stirling{n}{k}$ liczba sposobów podziału zbioru $n$ elementowego na $k$ nie pustych podzbiorów (również: liczba rozmieszczeń n rozróżnialnych obiektów w k identyczntych pudełkach, zakładając że pudełka nie mogą być puste).\\
\textbf{Definicja}\\
$\Stirling{n}{k}=\Stirling{n-1}{k-1}+k\Stirling{n-1}{k}, 0<k<n$ \\
$\Stirling{n}{n}=1, n\geq0$ $\Stirling{n}{0}=0, n>0$ \\
\textbf{Wzór jawny}: $S(n,k)=\Stirling{n}{k}=\frac{1}{k!}\sum_{i=0}^k(-1)^i{{k}\choose{k-i}}(k-i)^n$ \\
Wartości wygodnie wyznacza się korzsytając z trójkąta liczb stirlinga drugiego rodzaju:\\
\begin{tabular}{|r|r|r|r|r|r|r|r|r|r|} 
 \hline
 n\textbackslash k & 0 & 1 & 2 & 3 & 4 & 5 & 6 & 7 & 8\\ \hline
 0&1&&&&&&&& \\ \hline
 1&0&1&&&&&&&\\ \hline
 2&0&1&1&&&&&&\\ \hline
 3&0&1&3&1&&&&&\\ \hline
 4&0&1&7&6&1&&&&\\ \hline
 5&0&1&15&25&10&1&&&\\ \hline
 6&0&1&31&90&65&15&1&&\\ \hline
 7&0&1&\textbf{63}&\textbf{301}&350&140&21&1&\\ \hline
 8&0&1&127&\textbf{966}&1701&1050&266&28&1\\ \hline
\end{tabular}
np. $\Stirling{8}{3}=\Stirling{7}{2}+3\Stirling{7}{3}$\\
Interpretacje:\\
\begin{tabular}{c|c|c}
 Rozróżnialność k pudełek & Puste pudełka dopuszczalne & Liczba\\ \hline
 $-$ & $-$ & $S(n,k)$ \\
 $-$ & + & $\sum_{i=1}^k S(n,i)$ \\
 + & $-$ & $k! S(n,k)$ \\
 + & + & $\sum_{i=1}^k\frac{k!}{(k-i)!}S(n,i)$\\
\end{tabular}
\subsection{Liczby Bella}
Liczby Bella określają liczbę wszystkich podziałów zabioru $n$ elementowego: $B_n=\sum^n_{k=0}S(n,k)$
\subsection{Liczby Stirlinga pierwszego rodzaju}
$s(n,k)=\stirling{n}{k}$ oznacza liczbę sposobów rozmieszczenia $n$ obiektów w $k$ cyklach (orientacja cyklu jest istotna). \\
$\stirling{n}{k}=\stirling{n-1}{k-1}+(n-1)\stirling{n-1}{k}, 0<k<n$ \\
$\stirling{n}{n}=1, n\geq0$ $\stirling{n}{0}=0, n>0$ \\
Wartości wygodnie wyznacza się korzsytając z trójkąta liczb stirlinga pierwszego rodzaju:\\
\begin{tabular}{|r|r|r|r|r|r|r|r|r|} 
 \hline
 n\textbackslash k & 0 & 1 & 2 & 3 & 4 & 5 & 6 & 7\\ \hline
 0&1&&&&&&& \\ \hline
 1&0&1&&&&&&\\ \hline
 2&0&1&1&&&&&\\ \hline
 3&0&2&3&1&&&&\\ \hline
 4&0&6&11&6&1&&&\\ \hline
 5&0&24&50&35&10&1&&\\ \hline
 6&0&120&\textbf{274}&\textbf{225}&85&15&1&\\ \hline
 7&0&720&1764&\textbf{1624}&735&175&21&1\\ \hline
\end{tabular}
np. $\stirling{7}{3}=\stirling{6}{2}+6\stirling{6}{3}=274+6*255=1624$ \\
Suma każdego wiersza w trójkącie wynosi $n!$ \\
\textbf{Cykle} stanowią alternatywny zapis permutacji np.\\
\begin{tabular}{c|c|c|c|c|c|c|c|c|c}
 pozycja permutacji &1 & 2& 3& 4&5&6&7&8&9\\ \hline
 element & 3 & 8 &4 &7 &2 &9 &1 & 5 &6 \\
\end{tabular}
[1 3 4 7] [2 8 5] [6 9] \\
Każda permutacja definuje układ cykli i każdy układ cykli odpowiada dokładnie jednej permutacji.
\subsection{Liczby Eulera pierwszego rzędu}
$\left<  \begin{matrix} n \\ k\end{matrix} \right>$ oznacza liczbę permutacji zbioru $\{1,2,\dots, n\}$ mających k wzniesień, tzn. k pozycji na których $\pi_j<\pi_{j+1}$\\
$\left<  \begin{matrix} n \\ 0\end{matrix} \right>=1$,
$\left<  \begin{matrix} n \\ n\end{matrix} \right>=0$,
$\left<  \begin{matrix} n \\ k\end{matrix} \right>=(k+1)\left<  \begin{matrix} n-1 \\ k\end{matrix} \right> +(n-k) \left<  \begin{matrix} n-1 \\ k-1\end{matrix} \right>$ \\
\subsection{Liczby Eulera drugiego rzędu}
$\left<\left<  \begin{matrix} n \\ k\end{matrix} \right>\right>$ oznacza liczbę permutacji multizbioru $\{1,1,2,2,\dots,n,n\}$ takich, że między kolejnymi wystąpenieniami liczby $m$ pojawiają się tylko liczby większe od $m$ i liczba wzniesień wynosi $k$.\\
$\left<\left<  \begin{matrix} n \\ 0\end{matrix} \right>\right>=1$,
$\left<\left<  \begin{matrix} n \\ n\end{matrix} \right>\right>=0$,
$\left<\left<  \begin{matrix} n \\ k\end{matrix} \right>\right>=(k+1)\left<\left<  \begin{matrix} n-1 \\ k\end{matrix} \right>\right> +(2n-1-k) \left<\left<  \begin{matrix} n-1 \\ k-1\end{matrix} \right>\right>$ \\
\subsection{Liczby harmoniczne} %definicja: k-ta harmoniczna wytwarzana przez strunę skrzypcową jest tonem podstawowym wytwarzanym przez strunę o długości równej 1/k-tej długości struny wyjściowej (ton fali o długości  1/k jest k-tą harmoniczną fali o długości 1)
Definicja rekurencyjna: $\forall_{n\geq1}H_{n+1}=H_n+\frac{1}{n+1}, H_1=1$ \\
Definicja jawna: $H_n=\sum_{k=1}^n\frac{1}{k}$ \\
Liczby harmoniczne są dyskretnym odpowiednikiem $f(x)=ln(x)$ \\
$\forall_{n\in \mathbb{N}} H_{2^n}\leq1+n$ \\
$\forall_{n \in \mathbb{P}}\sum_{i=1}^n H_i=(n+1)H_n-n$
\subsection{Liczby Fibonacciego}
Definicja rekurencyjna: $\forall_{n\geq2} F_n=f_{n-1}+F_{n-2}, F_0=0, F_1=1$ \\
Definicja jawna: $F_n=\frac{1}{\sqrt{5}}[(\frac{1+\sqrt{5}}{2})^n - (\frac{1-\sqrt{5}}{2})^n]$
\subsection{Liczby Lucasa}
Definicja: $\forall_{n\geq2} L_n=L_{n-1} +L_{n-2}, L_0=2, L_1=1$\\
Własności: $\forall_{n\in\mathbb{P}} L_n=F_{n-1} +F_{n+1}$, $\sum_{i=0}^n l_i=L_{n+2}-1$
\subsection{Liczby Euklidesa}
$e_n=e_1\cdot e_2 \cdot \hdots \cdot e_{n-1} + 1$, $e_1=1$\\
Liczby euklidesa są względnie pierwsze $\gcd(e_n,e_m)=1$
\subsection{Liczby Mersennea}
$2^p-1$ gdzie $p$ jest liczbą pierwszą.\\
Rozwiązanie problemu wież z Hanoi (tzn. liczbę ruchów koniecznych do przełożenia $p$ krążków, dla $p$ będących liczbami pierwszymi).\\
Istnieje niewiele liczb Mersennea będących liczbami pierwszymi ''
Znanych jest 59\footnote{prof. Chris K. Caldwell \verb+https://primes.utm.edu/mersenne/index.html#known+} liczb pierwszych Mersennea, największa odpowiada wykładnikowi $p=82589933$ i ma $24 862 048$ cyfr\footnote{\verb+https://www.mersenne.org/primes/?press=M82589933+}

\section{Teoria grafów}
\textbf{Graf skierowany}: $G=(V,A)$ gdzie $V$-zbiór wierzchołków i $A\subseteq X\times X$-zbiór łuków.\\
\textbf{Graf nieskierowany}: $G=(V,E)$ gdzie $E\subseteq \{\{v_i,v_j\}:\}v_1\in V \wedge v_j\in V$-zbiór krawędzi.\\
Łuk (krawędź) $(x,y)\in A$ jest \textbf{incydentalny} z wierzchołkami $x$ i $y$.\\
\textbf{Pętla własna}: łuk (krawędź) postaci $(x,x)$, gdzie $x\in V$. \\
\textbf{Łańcuch}: sekwencja wierzchołków i krawędzi w grafu mająca końce w wierzchołkach $x, y$ i zawierającą $n$ krawędzi. Jeśli $x=y$ i $n>1$ to jest to \textbf{łańcuch zamknięty} w przeciwnym wypadku \textbf{łańcuch otwarty}.\\
\textbf{Droga}: łańcuch, w którym nie powtarzają się krawędzie (\textbf{obwód}-zamknięta droga).\\
\textbf{Ścieżka}: łańcuch, w którym nie powtarzają się wierzchołki (\textbf{cykl}-zamknięta ścieżka).\\
Jeżeli w grafie (nieskierowanym) istnieje droga między danymi wierzchołkami to istnieje też ścieżka między nimi.\\
\textbf{Podgraf}: $G_1=(V_1,E_1)$ jest podgrafem $G=(V,E)$ jeżeli: $\emptyset \neq V_1 \wedge E_1 \subseteq E$, gdzie każda krawędź ze zbioru $E_1$ jest incydentalna z wierzchołkami ze zbioru $V_1$\\
\textbf{Podgraf rozpinający}: jeśli $V_1=V$ \\
\textbf{Podgraf grafu $G$ indukowany zbiorem wierzchołków $U$}: podgraf, którego zbiorem wierzchołków jest $\emptyset\neq U\subseteq V$ i który zawiera wszystie krawędzie grafu $G$ postaci $\{x,y\}$ dla $x,y \in U$. \\
\textbf{Graf spójny}: posiada ścieżkę pomiedzy dowolnymi dwoma wierzchołkami.\\
Graf nieskierowany skojarzony z grafem skierowanym $G$, to taki graf w którym usunięto zwroty łuków, a następnie zduplikowane krawędzie. Jeśli graf skojarzony z grafem skierowanym jest spójny, to graf skierowany też jest spójny. \\
\textbf{Graf niespójny} to taki, który nie jest spójny.\\
Każdy graf niespójny składa się z dwóch lub więcej grafów spójnych. Każdy z tych podgrafów nazywamy \textbf{składową} grafu. Liczba składowych grafu $G$ oznaczana jest przez $\kappa(G)$. \\
\textbf{Multigraf}: dla pewnej pary wierzchołków istnieje więcej niż jeden łuk (krawędź).\\
\textbf{Graf pełny}: nieskierowany graf bez pętli własnych na zbiorze wierzchołków $V$ ($|V|=n$), w którym dla każdej pary $x,y\in V$ istnieje krawędź $\{x,y\}$. Oznaczany przez $K_n$.\\
\textbf{Dopełnienie} grafu $G$: $\bar G$ będący podgrafem $K_n$, zawierający wszystkie wierchołki $G$ i te krawędzie, których nie ma w $G$.
\subsection{Izomorfizm grafów}
Dwa grafy ($G_1=(V_1,E_1),G_2=(V_2,E_2)$ nazywamy \textbf{izomorficznymi} jeśli istnieje  $f: V_1 \rightarrow V_2$ (nazywana izomorfizmem grafów) że dla każdej pary wierzchołków $x,y \in V_1, \{x,y\}\in E_1 \Leftrightarrow \{f(x),f(y)\} \in E_2$

\subsection{Grafy planarne i dwudzielne}
\textbf{Graf planarny}: graf który można narysować na płaszyźnie w taki sposób, że żadne z jego łuków (krawędzi) nie przecinają się.\\
\textbf{Graf dwudzielny}: $G(V,E)$ gdzie $V=V_1\cup V_2$, $V_1\cap V_2=\emptyset$ oraz dla każdej krawędzi $\{x,y\}\in E$ zachodzi $x\in V_1 \wedge y\in V_2$. Oznaczamy $K_{m,n}$ gdzie $m=|V_1|, n=|V_2|$ (Każdy wierzchołek z $V_1$ jest połączony krawędzia z każdym wierchołkiem z $V_2$)\\
\textbf{Elementarny podział} grafu $G$ następuje, gdy z $G$ usuwamy krawędź $e=\{x,y\}$ i od grafu $G-e$ dodawane są krawędzie $\{x,u\}$ oraz $\{u,y\}$ (gdzie $u \notin V$).\\
Dwa grafy są \textbf{homeomorficzne} jeśli są izomorficzne lub jeżeli mogą zostać otrzymane z tego samego grafu (nieskierowanego, bez pętli własnych) w wyniku wykonania sekwencji elementarnych podziałów.\\
\textbf{Twierdzenie Kuratowskiego}: Graf nie jest planarny wtedy i tylko wtedy, gdy zawiera podgraf homeomorficzny do $K_3$ lub $K_{3,3}$.
\subsection{Grafy Eulera i Hamiltona}
$G=(V,E)$-(multi)graf nieskierowany\\
\textbf{Stopień} wierzchołka $v$ grafu $G$ ($deg(v)$) jest liczbą krawędz
i incydentalnych z $v$. (Pętlę własną traktujemy jako dwie krawędzie incydentalne z wierzchołkiem.)\\
$\sum_{v\in V} deg(v)=2|E|$ (wniosek: dla $G$ liczba wierzchołków o stopniu nieparzystym jest parzysta)\\

$G=(V,E)$-(multi)graf nieskierowany bez wierzchołków izolowanych\\
\textbf{Obwód Eulera} w grafie $G$ jest to obwód, który przechodzi przez każdą krawędź grafu $G$ dokładnie raz.\\
\textbf{Droga Eulera} otwarta droga między pewną parą wierzchołków która przechodzi przez każdą krawędź grafu $G$ dokładnie raz.\\
\textbf{Graf Eulera}: graf zawierający obwód Eulera\\
$G$ zawiera obwód Eulera $\Leftrightarrow$ $G$ jest spójny i każdy wierzchołek $G$ ma parzysty stopień. \\
$G$ zawiera drogę Eulera $\Leftrightarrow$ $G$ jest spójny i dokładnie dwa wierchołki mają nieparzysty stopień. \\
\textbf{Cykl Hamiltona} w grafie $G$ jest to cykl przechodzący przez każdy wierzchołek grafu dokładnie raz.\\
\textbf{Ścieżka Hamiltona} ścieżka przechodząca przez każdy wierzchołek grafu dokładnie raz.\\
\textbf{Graf Hamiltona}: Graf zawierający cykl hamiltona.
\subsection{Drzewa}
Graf nazywamy \textbf{drzewem} jeśli jest spójny i nie zawiera cykli.\\
\textbf{Las}: graf składający się z kilku składowych, z których każda jest drzewem.\\
\textbf{Drzewo rozpinające} grafu $G$: podgraf rozpinający, będący drzewem.\\
W drzewie miedzy każdą parą wierzchołków istnieje dokładnie jedna scieżka.\\
Graf nieskierowany jest spójny wtedy i tylko wtedy gdy posiada drzewo rozpinające.\\
Dla każdego drzewa $T=(V,E)$ zachodzi $|V|=|E|+1$.
%coś o korzeniach...
\section{Własności liczb całkowitych}
$b|a$ wtw. $\exists_n:a=bn$, $b$ jest \textbf{dzielnikiem} $a$, natomiast $a$ jest \textbf{wielokrotnością} $b$.\\
Dla dowolnych $a,b,c\in \mathbb{Z}$:
	\begin{enumerate}[a)]
		\item $1|a$ oraz $a|0$
		\item $[(a|b)\wedge(b|a)]\Rightarrow a=\pm b$
		\item $[(a|b)\wedge(b|c)]\Rightarrow a|c$
		\item $a|b \Rightarrow a|bx$ dla każdego $x\in\mathbb{Z}$
		\item jeżeli $x=y+z$ oraz $a$ dzieli dwie z tych liczb to dzieli również trzecią
		\item $[(a|b)\wedge(a|c)]\Rightarrow a|(bx+cy)$ ($bx+cy$ nazywamy liniową kombinacją $b$ i $c$)(zachodzi również dla wielu zmiennych w nawiasie)
	\end{enumerate}
\textbf{Liczby pierwsze}\\
\textbf{Algorytm dzielenia}: jeżeli $a, b \in \mathbb{Z}$ oraz $b>0$, to istnieją unikalne $q,r\in\mathbb{P}$ takie, że $a=qb+r$, $0\leq r <b$.
\subsection{Największy wspólny dzielnik}
	Dla $a, b\in \mathbb{Z}$ dodatnian liczba całkowita $c$ jest \textbf{wspólnym dzielnikiem} jeżeli $c|a$ oraz $c|b$.\\
	$c\in\mathbb{P}$ jest \textbf{największym wspólnym dzielnikiem} liczb $a, b$ jeżeli $c|a$, $c|b$ oraz dla dowolnego wspołnego dzielnika $d$ liczb $a, b$ zachodzi $d|c$.\\
	Liczby $a,b$ są \textbf{względnie pierwsze} jeżeli $\gcd(a,b)=1$\\
	\textbf{Algorytm euklidesa}$$
	\begin{matrix}
		a=q_1b+r_1 &0<r_1<b\\
		b=q_1r_1+r_2 & 0<r_2<r_1\\
		r_1=q_3r_2+r_3 & 0<r_3<r_2\\
		\vdots & \vdots \\
		r_i=q_{i+2}r_{i+1}+r_{i+2} & 0<r_{i+2}<r_{i+1}\\
		\vdots & \vdots \\
		r_{k-3}=q_{k-1}r_{k-2}+r_{k-1} & 0<r_{k-1}<r_{k-2}\\
		r_{k-2}=q_{k}r_{k-1}+r_{k} & 0<r_{k}<r_{k-1}\\
		r_{k-1}=q_{k+1}r_k
	\end{matrix}$$\\
	ostatnia niezerowa reszta $r_k=\gcd(1,b)$\\
\textbf{Rówananie diofantyczne}: równanie liniowe wymagające całkowitoliczbowych roziązań.\\
$ax+by=c$ ma rozwiązanie całkowite wtw. gdy $gcd(a,b)|c$
\subsection{Najmniejsza wspólna wielokrotność}
\textbf{wspólna wielokrotność}: $c$ jest wspólną wielokrotnością $a$ i $b$ jeśli $c$ jest wielokrotnością zarówno $a$ jak i $b$.\\
$c$ jest \textbf{najmniejszą wspólną wielokrotnością}, jeżeli jest najmniejszą z liczb będących wielokrotnością $a$ i $b$\\
$c=lcm(a,b)$ \\
$ab=lcm(a,b)\times\gcd(a,b)$
\subsection{Zasadnicze Twierdzenie Arytmetyki}
Jeżeli $a,b\in\mathbb{P}$ oraz $p$ jest liczbą pierwszą, to $p|ab\Rightarrow p|a \vee p|b$ (zachodzi również dla większej liczby czynników).\\
\textbf{Zasadnicze Twierdzenie Arytmetyki}: Każda liczba całkowita $n>1$ może zostać zapisana jako iloczyn liczb pierwszych, przy czym istnieje tylko jeden taki iloczyn.
\section{Transwersale}
\subsection{Twierdzenie Halla-wersja małżeńska}W grupie kobiet każda kobiera może wybrac męża spośród mężczyzn, których zna, wtw. gdy w każdym podzbiorze kobiet o liczności $r$ kobiety te znają conajmniej $r$ mężczyzn.\\
\textbf{Rodzina zbiorów} jest to uporządkowana lista zbiorów $A=(A_1,A_2,A_3,\hdots,A_n)$\\
\textbf{Transwersala} rodziny $A$ jest to zbiór $Y$, gdzie $|Y|=n$, którego elementy mogą zostać zapisane w pewnym porządku, np. $Y=\{ a_1, a_2, a_3,\hdots, a_n\}$ w ten sposób że $a_i\in A_i$.
\subsection{Turnieje}
\textbf{Turniej} z udziałem $n$ graczy oanzacoznych kolejenymi liczbami pozytywnymi, jest zbiorem $n\choose2$ uporządkowanych par zawodników. Pary te mogą być traktowane jako wyniki gier między graczami, przy czym pierwsza liczba oznacza zwycięzce gry.\\
\textbf{Twierdzenie Landaua}: Niech $b_1, b_2,\hdots,b_n$ będą liczbami całkowitymi. Liczby te są wynikiem pewnego turnieju z udziałem $n$ graczy wtw. gdy $b_1+b_2+\hdots+b_n={n\choose2}$ oraz dla $1\leq r \leq n$ każde $r$ spośród liczb $b_1, b_2,\hdots,b_n$ sumuje się do co najmniej $r\choose2$
%minimaks, harem, przepływy, przekroje itp...
\section{Zaawansowane metody zliczania}
\subsection{Zasada włączania i wyłączania}
	$|A_1\cup A_2| = |A_1| + |A_2| -|A_1\cap A_2|$ \\
	$|A_1\cup A_2\cup A_3| = |A_1| + |A_2| + |A_3|-|A_1\cap A_2| -|A_1\cap A_3| -|A_2\cap A_3|+ |A_1\cap A_2\cap A_3|$\\
	Aby znaleźć liczbę elementów zbioru $|A_1\cap A_2\cap \hdots \cap A_n|$ należy zanalizować wszystkie możliwe przecięcia (części wspólne) zbiorów z rodziny $(A_1, A_2, \dots ,A_n)$ i dodać liczność przecięć nieparzystej liczby zbiorów oraz odjąć liczność przecięć nieparzystej liczby zbiorów.\\
	$\left| \bigcup\limits_{i=1}^n A_i\right|=\sum\limits_{I\in\mathbb{P}_+(I)} (-1)^{|I|+1} \left| \bigcap\limits_{i\in I} A_i \right|$
\subsection{Zasada szufladkowa Dirichleta}
	Jeśli $m$ przemiotów umieszcza się w $n$ szufladkach i $m>n$, to do jednej szufladki trafi więcej niż jeden przedmiot.\\
	Jeśli skońcozny zbiór $S$ jest podzielony na $k$ rozłącznych niepustych podzbiorów, to co najmniej jeden z tych podzbiorów ma $\frac{|S|}{k}$ elementów lub więcej.
	%uogólniona
\subsection{Zasada dwoistości}
	Zasada dwoistości nie jest formalną zasadą, ale często stosowaną metodą dowodzenia.\\
	Zasada umożliwia wykluczanie pewnych sytuacji dzięki wykorzystaniu przeciwstawności (dwoistości) np.:
	\begin{itemize}
		\item białe - czarne 
		\item parzyste - nieparzyste
	\end{itemize}
\section{Kwadraty łacińskie}
\section{Funkcje tworzące}
\section{Wielomiany szachowe}

\end{document}
